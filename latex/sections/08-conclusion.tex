\section{Conclusion: From Assistants to True Partners}
\label{sec:conclusion}

Agentic drug discovery has reached a genuine inflection point. Systems like ChatInvent, Coscientist, and ChemCrow show that LLMs can orchestrate tools, navigate literature, and generate hypotheses. But current architectures are optimized for small-molecule discovery at well-resourced pharmaceutical companies, and blind spots appear outside that envelope.

This paper identified five gaps through practitioner experience across multiple projects: small-molecule bias limiting peptide and biologic work; absence of in vivo to in silico bridges for longitudinal, multi-modal data; LLM-centric orchestration excluding ML training, RL, simulation, and constrained optimization; large-pharma resource assumptions ignoring small biotech realities; and single-metric optimization missing multi-objective trade-offs under uncertainty. Addressing these gaps requires multi-paradigm orchestration, modality-aware architectures, in vivo integration, data-efficient learning, and Pareto-based optimization with uncertainty quantification.

These systems should not replace human expertise; drug discovery is too complex and context-dependent. The better framing is computational partners that handle preprocessing, model training, tuning, and visualization while practitioners focus on hypotheses, mechanistic interpretation, and strategic trade-offs. Partnership requires bidirectional learning: agents explain assumptions and uncertainty, practitioners provide feedback, and both improve over time.

Success is practitioner impact: faster lead identification, reduced experimental waste, and better decisions under uncertainty. If next-generation systems embody these principles, they can transform workflows. If they remain LLM-centric tools designed for large pharma, they will not scale. The difference is architectural, not cosmetic, and it will determine whether agents become core infrastructure or remain demonstrations. That choice will be visible in how teams allocate compute, structure experiments, and trust model outputs across discovery stages.

\subsection{A Call to Action}

\textbf{For researchers:} Move beyond tool-calling to multi-paradigm orchestration. Build data-efficient, modality-aware systems with uncertainty quantification and Pareto visualization. Prioritize batch-mode workflows over interactive chat for practitioners managing multiple projects.

\textbf{For practitioners:} Share anonymized problem formulations and dataset characteristics. Demand systems that respect resource constraints, quantify uncertainty, and support human-in-the-loop decision-making. Provide feedback on where agent recommendations succeed or fail.

\textbf{For funders and organizations:} Support open infrastructure for PLMs, structural biology tools, reproducible datasets, and workflow orchestration. Prioritize translational impact and cross-sector collaborations between large pharma and resource-constrained biotechs.

Comprehensive surveys map the landscape of existing tools \citep{seal2025aiagents} and calls for adoption are necessary \citep{lakhan2025agentic}, but neither addresses the architectural gaps that prevent these systems from working in most real-world settings. This paper complements both: the gaps we identify are precisely the ones the surveys catalog but do not critique, and the adoption Lakhan advocates requires the engineering solutions we propose.

Progress since 2023 showed what is possible. The next three years will determine whether these systems generalize beyond demos. The gaps are engineering challenges. If the field meets them, agentic systems can become core infrastructure for drug discovery. If not, adoption will remain limited.
