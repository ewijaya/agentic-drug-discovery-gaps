\subsection{Gap 2: Absence of In Vivo-In Silico Integration}
\label{sec:invivo}

The absence of in vivo modeling is a fundamental gap. Current agents excel at in vitro automation: Coscientist plans syntheses \citep{boiko2023coscientist}, ChemCrow screens compound libraries, ChatInvent mines literature. But critical validation happens in vivo, where candidates confront pharmacokinetics, biodistribution, metabolism, toxicology, and long-term efficacy unpredictable from binding affinity.

Animal studies generate a distinct class of data: longitudinal (days to months), multi-modal (behavioral scores, imaging, molecular profiling), noisy (biological variability dwarfs plate assay precision), low-throughput (tens of compounds, not thousands), and expensive. These characteristics make in vivo the bottleneck, yet agents provide no pathway to incorporate this data.

\subsubsection{Findings}

Task classes 4, 9, 12, and 13 (in vivo and imaging tasks) receive zero coverage across all six frameworks. All frameworks terminate at in vitro automation or literature-based hypothesis generation. No framework supports longitudinal data modeling, multi-modal fusion, or causal inference from animal study data.

\subsubsection{The Lab Automation Ceiling}

Lab automation reaches a hard ceiling at in vivo studies. Synthesis platforms and high-throughput screening test thousands of compounds daily. But animal experiments cannot be scaled or automated, requiring specialized facilities, personnel, ethical oversight, and weeks of time.

An agent might design a peptide and predict binding, but cannot integrate a 28-day traumatic brain injury study measuring behavioral recovery, histological regeneration, and transcriptomic neuroprotection. Data formats, temporal structure, and statistical requirements exceed current capabilities.

In vivo studies yield heterogeneous streams. Neurological injury evaluation includes behavioral assessments (motor coordination via beam walking, generating ordinal scores), tissue histology (cell proliferation requiring computer vision), RNA sequencing (high-dimensional gene expression needing differential expression and pathway enrichment), and clinical notes (semi-structured weight and adverse events). This multi-modal integration remains outside current agent scope.

Developing composite efficacy metrics exposed this gap. A peptide increasing cell proliferation threefold in vitro shows temporal in vivo dynamics: inflammatory response (days 1-3), progenitor proliferation (days 7-10), functional recovery (day 28). Predicting sustained benefit requires temporal modeling, dataset curation, feature engineering, and validation outside LLM tool-calling scope.

\subsubsection{Multi-Modal, Longitudinal Data Integration}

\begin{figure}[htbp]
\centering
\includegraphics[width=\textwidth]{figures/fig3-invivo-insilico-gap.jpeg}
\caption{What Agents Can and Cannot Process. Matrix showing data types on the vertical axis and agent accessibility on the horizontal axis. Green checkmarks indicate data types current agents handle well (text, SMILES, PDB files, CSV data, literature). Red X marks denote data types agents struggle with (behavioral videos, clinical score trajectories, tissue imaging, multi-modal transcriptomics, longitudinal measurements with dropout). This visualization reveals the systematic exclusion of in vivo data modalities from current agent architectures.}
\label{fig:invivo-gap}
\end{figure}

The core challenge is that in vivo data does not fit the tidy CSV format that machine learning pipelines expect (Table~\ref{tab:data-accessibility}). Behavioral scores are ordinal and subject to inter-rater variability.

\begin{table}[htbp]
\centering
\caption{Data Type Accessibility for Current Agent Systems}
\label{tab:data-accessibility}
\begin{tabular}{lllp{4cm}}
\hline
\textbf{Data Type} & \textbf{Format} & \textbf{Agent-Readable} & \textbf{Example Use Case} \\
\hline
SMILES strings & Text & Yes & Small molecule property prediction \\
Literature abstracts & Text & Yes & Knowledge synthesis \\
PDB structures & Structured & Yes & Protein structure analysis \\
CSV assay data & Tabular & Yes & High-throughput screening \\
Behavioral videos & Video & No & Phenotyping analysis \\
Clinical trajectories & Time-series & Partial & Longitudinal efficacy modeling \\
Tissue imaging & Image & Partial & Histological quantification \\
RNA-seq data & FASTQ/BAM & No & Transcriptomic profiling \\
Clinical notes & Semi-structured text & Partial & Adverse event detection \\
\hline
\end{tabular}
\end{table}

Behavioral phenotyping via DeepLabCut \citep{mathis2018deeplabcut} tracks animal poses in videos, generating time-series keypoint coordinates. Computing behavioral metrics (inter-animal distance, contact time, grooming) requires training pose estimation, validating tracking, computing features, and statistical testing. Practitioners must handle this workflow manually, spanning video processing, supervised learning, time-series engineering, and hypothesis testing.

RNA-seq requires quality control, alignment, and quantification into expression matrices. Differential expression identifies treatment effects. Pathway enrichment maps genes to biological processes via KEGG \citep{kanehisa2023kegg} or Gene Ontology. Upstream regulator analysis infers transcription factors driving changes. The FASTQ-to-hypothesis pipeline needs bioinformatics tools (STAR, HISAT2, DESeq2, edgeR, GSEA \citep{subramanian2005gsea}) that agents do not integrate. Recent systems like Medea \citep{medea2026} have begun addressing multi-omics analysis agenically, handling transcriptomics, protein networks, and pathway analysis. However, these process static datasets rather than longitudinal in vivo time-series, and do not integrate behavioral phenotyping, imaging, or temporal efficacy modeling.

Integrating heterogeneous sources for predictive biomarkers is valuable yet absent. Correlating in vitro bioactivity with in vivo efficacy required extracting features from multiple assays, normalizing across scales, aligning with temporal data (days 3, 7, 14, 28), and training regression models predicting long-term outcomes. This workflow involved feature engineering, imputation, stratified cross-validation, and model selection; these are ML workflows, not LLM reasoning.

\subsubsection{Safety, Efficacy, and Translation}

In vivo models surface safety-efficacy trade-offs that in silico screens miss. Tenfold bioactivity increases may trigger immune activation or hepatotoxicity. Stability modifications (D-amino acids, cyclization) may reduce affinity or increase aggregation.

Toxicology requires dose-response analysis via generalized linear mixed models accounting for repeated measures and time-dependent effects. Identifying therapeutic windows where efficacy plateaus but toxicity remains acceptable is essential. Species translation compounds this: mouse pharmacokinetics extrapolate to humans via allometric scaling, but peptide stability varies by species protease expression and rodent-tolerated peptides may provoke primate antibody responses. Agents cannot construct dose-response curves, compute LD50 confidence intervals, or model cross-species translation uncertainties. We return to the broader multi-objective trade-offs these challenges imply in \S\ref{sec:multiobjective}.

\subsubsection{Derived Requirements}

Closing the in vivo-in silico gap requires three capabilities absent from current frameworks. First, temporal state-space models for longitudinal in vivo trajectories that capture treatment dynamics across days to weeks. Second, causal inference tools (do-calculus, counterfactual reasoning) to separate correlation from mechanism in complex biological systems. Third, multi-modal data fusion integrating clinical scores, imaging, transcriptomics, and behavioral data into unified predictive models. Until agents ingest longitudinal scores, integrate transcriptomics, quantify dose-response uncertainty, and navigate multi-objective trade-offs under biological variability, utility remains confined to early hit identification. Most development cost and risk lies in translating in vitro activity to in vivo efficacy and safety \citep{dimasi2016costs}.
