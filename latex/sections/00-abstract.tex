Agentic AI systems like ChatInvent and Coscientist demonstrate impressive capabilities but reveal systematic blind spots when applied beyond small-molecule workflows at well-resourced pharmaceutical companies. Drawing on experience leading 14+ AI-driven drug discovery projects at a small biotech specializing in therapeutic peptides, this perspective identifies five critical gaps: the small molecule bias excluding protein language models and peptide-specific property prediction; the absence of in vivo to in silico bridges for longitudinal, multi-modal animal data; LLM-centric orchestration that excludes machine learning training, reinforcement learning, and multi-paradigm coordination; resource-abundant design assumptions mismatched to small biotech realities of limited data and modest compute; and single-metric optimization ignoring multi-objective trade-offs in safety, efficacy, and stability. Each gap is illustrated through practitioner experience spanning peptide generation, behavioral phenotyping, multi-endpoint prediction, and longitudinal in vivo modeling. We propose five design principles: multi-paradigm orchestration treating ML training and RL as architectural primitives, modality-aware architectures supporting peptides and biologics, in vivo integration through temporal modeling and causal inference, data-efficient learning via transfer and active learning, and multi-objective optimization with uncertainty quantification and Pareto visualization. The goal is building computational partners that augment practitioner judgment, not replacing human expertise, and that can operate under realistic data, modality, and resource constraints.
