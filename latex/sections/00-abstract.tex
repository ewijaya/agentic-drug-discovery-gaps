The emergence of agentic AI systems in drug discovery, from ChatInvent's deployment at AstraZeneca to autonomous synthesis platforms like Coscientist, signals a transformative shift in computational approaches to therapeutic development. However, these systems reveal systematic blind spots when applied beyond their design context of small-molecule, target-based workflows at well-resourced pharmaceutical companies. Drawing on experience leading 14+ AI-driven drug discovery projects at a small biotech specializing in therapeutic peptides and regenerative medicine, this perspective identifies five critical gaps that limit the practical utility of current agent architectures: the small molecule bias in tool design and model integration; the absence of in vivo to in silico bridges for longitudinal, multi-modal animal study data; the LLM-centric orchestration paradigm that excludes gradient-based optimization, reinforcement learning, and multi-paradigm workflow coordination; the mismatch between resource-abundant design assumptions and small biotech realities of limited data, modest compute budgets, and lean teams; and the single-metric optimization focus that ignores multi-objective trade-offs inherent to safety, efficacy, and stability navigation. Each gap is illustrated through anonymized practitioner experience spanning peptide generation with protein language models, behavioral phenotyping from video data, composite efficacy metric development for traumatic brain injury models, and multi-endpoint bioactivity prediction. We propose five design principles for next-generation agents: multi-paradigm orchestration supporting machine learning training and reinforcement learning as architectural primitives, modality-aware architectures with first-class support for peptides and biologics, in vivo to in silico integration through temporal modeling and causal inference, data-efficient learning via transfer learning and active learning loops, and multi-objective optimization with uncertainty quantification and Pareto frontier visualization. The goal is not to replace human expertise but to build computational partners that augment practitioner judgment in navigating the irreducible complexity of biological systems.
