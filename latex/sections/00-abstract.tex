Agentic systems have advanced drug discovery automation but reveal systematic blind spots beyond small-molecule workflows at well-resourced pharmaceutical companies. Drawing on experience with computational drug discovery projects at a small biotech specializing in therapeutic peptides, this perspective identifies five critical gaps: small molecule bias excluding protein language models and peptide-specific prediction; absent in vivo to in silico bridges for longitudinal, multi-modal animal data; LLM-centric orchestration that excludes ML training, reinforcement learning, and multi-paradigm coordination; resource assumptions mismatched to small biotech realities; and single-metric optimization ignoring multi-objective trade-offs in safety, efficacy, and stability. We propose design principles addressing each gap, including multi-paradigm orchestration, modality-aware architectures, in vivo integration, data-efficient learning, and Pareto-based optimization with uncertainty quantification. The goal is computational partners that augment practitioner judgment under realistic data, modality, and resource constraints.
