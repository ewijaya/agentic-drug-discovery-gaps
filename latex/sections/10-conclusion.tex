\section{Conclusion}
\label{sec:conclusion}

We evaluated six agentic frameworks against 15 drug discovery task classes and identified five critical capability gaps: small-molecule representation bias, absence of in vivo-in silico integration, limited computational paradigm support, misalignment with small-biotech constraints, and single-objective optimization assumptions. These gaps reflect architectural constraints that individual feature additions do not address: closing them requires changes to core framework design.
Our knowledge probing experiment reinforces this conclusion: four frontier LLMs
demonstrate competent peptide reasoning across all categories tested, yet this expertise
remains stranded behind agent architectures that provide no peptide-aware tools or
sequence-native workflows.

From the identified gaps, we derived five design requirements for next-generation frameworks: multi-paradigm orchestration supporting ML training, RL, and simulation as first-class primitives; modality-aware representations for peptides, proteins, and biologics; in vivo data integration with temporal modeling and multi-modal fusion; data-efficient learning through few-shot adaptation, active learning, and transfer learning; and multi-objective, risk-aware optimization with Pareto frontiers and uncertainty quantification.

The capability matrix and derived requirements provide a roadmap for framework developers and a benchmarking tool for practitioners evaluating agentic systems against their specific discovery contexts. Progress since 2023 has established what agentic systems can achieve. The next phase will determine whether these systems generalize beyond small-molecule workflows at well-resourced organizations to become core infrastructure for drug discovery broadly.
