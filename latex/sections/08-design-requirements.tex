\section{Design Requirements for Next-Generation Frameworks}
\label{sec:requirements}

The gaps identified in the preceding analysis reflect architectural assumptions that limit current frameworks. This section synthesizes five design requirements derived from the gap analysis and illustrates them with concrete use cases.

\subsection{Requirements Derived from Gap Analysis}

Table~\ref{tab:gap-requirements} summarizes the five design requirements derived from the gap analysis. Each requirement addresses the architectural limitations identified in the corresponding gap section, where detailed evidence and practitioner context are provided.

\begin{table}[htbp]
\centering
\caption{Design Requirements Derived from Gap Analysis}
\label{tab:gap-requirements}
\small
\begin{tabular}{p{3.5cm}p{4cm}p{4cm}}
\hline
\textbf{Requirement (Gap)} & \textbf{Core Capabilities} & \textbf{Key Primitives} \\
\hline
R1: Multi-paradigm orchestration (Gap~3) & ML training, RL, simulation, and optimization as first-class primitives & Declarative workflow graphs, checkpointing, human-in-the-loop decision points \\[4pt]
R2: Modality-aware representations (Gap~1) & PLM fine-tuning, structural biology integration, peptide-specific prediction & ESM-2/ProtBERT pipelines, AlphaFold integration, aggregation prediction \\[4pt]
R3: In vivo--in silico data fusion (Gap~2) & Temporal modeling, multi-modal fusion, causal inference & State-space models, bioinformatics pipelines, pathway enrichment \\[4pt]
R4: Data-efficient learning (Gap~4) & Few-shot adaptation, active learning, transfer learning & Meta-learning, Bayesian uncertainty, public-to-private fine-tuning \\[4pt]
R5: Risk-aware optimization (Gap~5) & Pareto optimization, uncertainty quantification, sensitivity analysis & Constraint handling, frontier visualization, stage-adaptive recommendations \\
\hline
\end{tabular}
\end{table}

\subsection{Illustrative Use Cases}

These requirements imply a shift from chat-first tooling to workflow-first systems. Agents would maintain state across iterations, log decisions, and make assumptions explicit so practitioners can audit results and reproduce outcomes. This is a standard requirement in regulated environments and for teams that revisit decisions months later, often under new personnel or budget constraints.

To make these requirements concrete, we describe three representative use cases that current agents cannot handle but next-generation systems should support.

\subsubsection{Use Case 1: Peptide Lead Optimization}

\textbf{Input:} Fifty peptides with four assay endpoints, receptor structure, synthesis constraints.

\textbf{Workflow:} Fine-tune ESM-2, train multi-task regressor, use it as RL reward, filter by synthesis feasibility and stability, dock top candidates, cluster binding modes, present activity versus safety Pareto frontier with uncertainty.

\textbf{Output:} Ten synthesis candidates with rationales and confidence intervals.

\textbf{Human decisions:} Select among Pareto-optimal candidates based on strategic priorities and budget.

\subsubsection{Use Case 2: In Vivo Efficacy Prediction}

\textbf{Input:} In vitro assays and early in vivo markers for 20 peptides; predict day 28 outcomes.

\textbf{Workflow:} Normalize features, align temporal data with missing values, train regression models with stratified validation, identify early predictors and mechanistic drivers.

\textbf{Output:} Day 28 predictions with uncertainty and mechanistic hypotheses.

\textbf{Human decisions:} Choose peptides for full validation and whether to collect additional early markers.

\subsubsection{Use Case 3: Multi-Endpoint Assay Analysis}

\textbf{Input:} Four-endpoint data for 100 peptides.

\textbf{Workflow:} Normalize, cluster, validate stability, extract enriched sequence motifs, run pathway enrichment, visualize clusters.

\textbf{Output:} Distinct activity profile clusters with mechanistic hypotheses and selection recommendations.

\textbf{Human decisions:} Validate hypotheses and decide whether to focus on a single cluster or maintain diversity.

\subsection{Infrastructure Considerations}

\begin{table}[htbp]
\centering
\caption{Gap-to-Requirement Mapping: Priority Matrix for Next-Generation Agent Features}
\label{tab:requirement-priority}
\small
\begin{tabular}{lcccp{3cm}}
\hline
\textbf{Feature} & \textbf{Impact} & \textbf{Difficulty} & \textbf{Who Needs It} & \textbf{Priority} \\
\hline
Multi-paradigm orchestration & High & High & All & Critical \\
PLM fine-tuning pipelines & High & Medium & Biologics & Critical \\
Active learning loops & High & Medium & Small biotech & Critical \\
Pareto frontier visualization & High & Medium & All & High \\
Uncertainty quantification & High & Medium & All & High \\
In vivo data integration & High & High & All & High \\
Batch-mode workflows & Medium & Low & Small biotech & Medium \\
Transfer learning support & High & Medium & Small biotech & High \\
Constraint satisfaction & Medium & Medium & All & Medium \\
Workflow checkpointing & Low & Low & All & Medium \\
\hline
\end{tabular}
\end{table}

Realizing these capabilities requires infrastructure beyond current agents. API standards would need to cover PLMs (embedding extraction, fine-tuning, uncertainty), structural biology tools (AlphaFold, docking, molecular dynamics), and bioinformatics pipelines (alignment, differential expression, pathway enrichment). Version-controlled datasets, model registries, and provenance tracking are standard requirements for reproducibility and auditability.

Organizational alignment matters. Systems targeting small biotech contexts would need to fit modest compute budgets, minimal storage, and lightweight deployment. Documentation targeting non-ML-expert biologists would lower adoption barriers. Integration with existing tools (GraphPad, FlowJo, ImageJ, R/Bioconductor) avoids workflow disruption.
