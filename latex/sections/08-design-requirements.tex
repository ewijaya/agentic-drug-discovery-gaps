\section{Design Requirements for Next-Generation Frameworks}
\label{sec:requirements}

The gaps identified in the preceding analysis reflect architectural assumptions that limit current frameworks. This section synthesizes five design requirements derived from the gap analysis and illustrates them with concrete use cases.

\subsection{Requirements Derived from Gap Analysis}

\subsubsection{Requirement 1: Multi-Paradigm Orchestration}

Closing Gap 3 requires support for ML training, RL, simulation, and optimization as first-class primitives. Effective frameworks would allow practitioners to specify workflows declaratively and receive an executable workflow graph supporting parallelization, checkpointing, and branching when validation fails. Human-in-the-loop decision points need to be explicit when models trade off accuracy, calibration, or interpretability. Workflow orchestration systems (Airflow, Kubeflow, Nextflow) already provide these abstractions; what is missing is tight integration with agent reasoning and iterative improvement.

\subsubsection{Requirement 2: Modality-Aware Representations}

Closing Gap 1 requires first-class support for peptides, proteins, and other modalities, not just small molecules. This entails PLM fine-tuning pipelines, structural biology integration (AlphaFold, flexible docking, molecular dynamics), and peptide-specific property prediction (aggregation, protease resistance, immunogenicity). Modality awareness would flow through data loaders, feature extractors, generative models, and evaluation metrics. Contextual tool selection would default peptides to ESM-2 embeddings and protein structures to structural alignment, rather than SMILES-based similarity.

\subsubsection{Requirement 3: In Vivo-In Silico Data Fusion}

Closing Gap 2 requires temporal modeling for longitudinal efficacy and safety data, multi-modal fusion (behavior, imaging, transcriptomics), and causal inference for mechanistic hypotheses. This entails bioinformatics pipelines (RNA-seq alignment, differential expression, pathway enrichment), computer vision, and statistical models for dose-response and mixed effects. Predictive efficacy modeling would align heterogeneous timepoints, handle missing data, and validate on temporally ordered splits. Mechanistic inference would integrate KEGG, GO, and Reactome to translate results into hypotheses and validation experiments.

\subsubsection{Requirement 4: Data-Efficient Learning}

Closing Gap 4 requires optimization for few-shot adaptation, active learning, and knowledge transfer across related tasks. This entails meta-learning, Bayesian uncertainty quantification, and transfer pipelines combining public pre-training (UniProt, PDB, ChEMBL) with private fine-tuning. Effective frameworks would recommend strategies based on dataset size: deep nets for 500 examples, simpler models for 50, few-shot for 10. Active learning loops would select candidates via acquisition functions, update models with experimental feedback, and signal when marginal information gain is low.

\subsubsection{Requirement 5: Multi-Objective, Risk-Aware Optimization}

Closing Gap 5 requires Pareto optimization with constraints, uncertainty quantification, sensitivity analysis, and interactive trade-off visualization. Results would take the form of Pareto frontiers with confidence intervals, constraint status, and robustness scores. Practitioners would filter by feasibility, adjust objective weights, and highlight low-uncertainty candidates. Risk-aware systems would adapt recommendations to stage: early exploration versus late-stage de-risking.

\subsection{Illustrative Use Cases}

These requirements imply a shift from chat-first tooling to workflow-first systems. Agents would maintain state across iterations, log decisions, and make assumptions explicit so practitioners can audit results and reproduce outcomes. This is a standard requirement in regulated environments and for teams that revisit decisions months later, often under new personnel or budget constraints.

To make these requirements concrete, we describe three representative use cases that current agents cannot handle but next-generation systems should support.

\subsubsection{Use Case 1: Peptide Lead Optimization}

\textbf{Input:} Fifty peptides with four assay endpoints, receptor structure, synthesis constraints.

\textbf{Workflow:} Fine-tune ESM-2, train multi-task regressor, use it as RL reward, filter by synthesis feasibility and stability, dock top candidates, cluster binding modes, present activity versus safety Pareto frontier with uncertainty.

\textbf{Output:} Ten synthesis candidates with rationales and confidence intervals.

\textbf{Human decisions:} Select among Pareto-optimal candidates based on strategic priorities and budget.

\subsubsection{Use Case 2: In Vivo Efficacy Prediction}

\textbf{Input:} In vitro assays and early in vivo markers for 20 peptides; predict day 28 outcomes.

\textbf{Workflow:} Normalize features, align temporal data with missing values, train regression models with stratified validation, identify early predictors and mechanistic drivers.

\textbf{Output:} Day 28 predictions with uncertainty and mechanistic hypotheses.

\textbf{Human decisions:} Choose peptides for full validation and whether to collect additional early markers.

\subsubsection{Use Case 3: Multi-Endpoint Assay Analysis}

\textbf{Input:} Four-endpoint data for 100 peptides.

\textbf{Workflow:} Normalize, cluster, validate stability, extract enriched sequence motifs, run pathway enrichment, visualize clusters.

\textbf{Output:} Distinct activity profile clusters with mechanistic hypotheses and selection recommendations.

\textbf{Human decisions:} Validate hypotheses and decide whether to focus on a single cluster or maintain diversity.

\subsection{Infrastructure Considerations}

\begin{table}[htbp]
\centering
\caption{Gap-to-Requirement Mapping: Priority Matrix for Next-Generation Agent Features}
\label{tab:requirement-priority}
\small
\begin{tabular}{lcccp{3cm}}
\hline
\textbf{Feature} & \textbf{Impact} & \textbf{Difficulty} & \textbf{Who Needs It} & \textbf{Priority} \\
\hline
Multi-paradigm orchestration & High & High & All & Critical \\
PLM fine-tuning pipelines & High & Medium & Biologics & Critical \\
Active learning loops & High & Medium & Small biotech & Critical \\
Pareto frontier visualization & High & Medium & All & High \\
Uncertainty quantification & High & Medium & All & High \\
In vivo data integration & High & High & All & High \\
Batch-mode workflows & Medium & Low & Small biotech & Medium \\
Transfer learning support & High & Medium & Small biotech & High \\
Constraint satisfaction & Medium & Medium & All & Medium \\
Workflow checkpointing & Low & Low & All & Medium \\
\hline
\end{tabular}
\end{table}

Realizing these capabilities requires infrastructure beyond current agents. API standards would need to cover PLMs (embedding extraction, fine-tuning, uncertainty), structural biology tools (AlphaFold, docking, molecular dynamics), and bioinformatics pipelines (alignment, differential expression, pathway enrichment). Version-controlled datasets, model registries, and provenance tracking are standard requirements for reproducibility and auditability.

Organizational alignment matters. Systems targeting small biotech contexts would need to fit modest compute budgets, minimal storage, and lightweight deployment. Documentation targeting non-ML-expert biologists would lower adoption barriers. Integration with existing tools (GraphPad, FlowJo, ImageJ, R/Bioconductor) avoids workflow disruption.
